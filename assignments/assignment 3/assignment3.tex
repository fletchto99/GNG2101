\documentclass[fleqn, 12pt]{article}

% packages %
\usepackage[includeheadfoot,headheight=15pt,margin=0.5in,left=1in,right=1in,headsep=10pt]{geometry} % page margins %
\usepackage{mathtools, amssymb} % math %
\usepackage{tabularx, multirow} % tables %
\usepackage[outputdir=.latexcache]{minted} % code %
\usepackage{graphicx} % graphics %
\usepackage{enumerate} % lists %
\usepackage{adjustbox} % images %
\usepackage[T1]{fontenc} % fonts %
\usepackage[protrusion=true,expansion=true]{microtype} % font clarity %
\usepackage{fancyhdr} % headers and footers %
\usepackage{lastpage} % reference page numbers %
\usepackage{color} % colors for code %
\usepackage{tikz} % for graphs %
\usepackage{soul} % for strikethroughout %
\usepackage{upquote} % Fix single quotes %
\usepackage{etoolbox} % Conditional checks %
\usepackage[hidelinks]{hyperref} % Hyperlinks %
\usepackage{indentfirst} % fix indentation - only for essays %
\usepackage[figure,table]{totalcount} % For counting tables and figures %
\usepackage[utf8]{inputenc} % Encode as UTF-8 %
\usepackage{biblatex} % References %
\addbibresource{references.bib} % bib source %

% Document details %
\newcommand{\university}{University of Ottawa}
\newcommand{\name}{Matthew Langlois}
\newcommand{\studentNumber}{7731813}
\newcommand{\semester}{Winter 2018}
\newcommand{\assignmentType}{Assignment}
\newcommand{\assignmentNumber}{3}
\newcommand{\dueDate}{Jan 26, 2018}
\newcommand{\courseCode}{GNG2101}
\newcommand{\courseTitle}{Introduction to Product Development}
\newcommand{\essayTitle}{<Title>} % only for essays %
\newcommand{\essaySubtitle}{<subtitle>} % only for essays %
\newcommand{\essayAbstract}{} % Only for essays -- leave empty for no abstract %

% Center image and diagrams %
\adjustboxset*{center}

% Code settings %
\setminted{
    fontfamily=tt,
    gobble=0,
    frame=single,
    funcnamehighlighting=true,
    tabsize=4,
    obeytabs=false,
    mathescape=false
    samepage=false,
    showspaces=false,
    showtabs =false,
    texcl=false,
    breaklines=true,
}

% inline code %
\definecolor{codegray}{gray}{0.9}
\newcommand{\code}[2]{\colorbox{codegray}{\mintinline{#1}{#2}}}

% Code from tile %
\newcommand{\codefile}{\inputminted}

% Graphing stuff %
\usetikzlibrary{arrows.meta}
\usetikzlibrary{positioning}
\usetikzlibrary{matrix}
\usetikzlibrary{automata}

% Define ceiling and floor functions %
\DeclarePairedDelimiter\ceil{\lceil}{\rceil}
\DeclarePairedDelimiter\floor{\lfloor}{\rfloor}

% Create set compliment command %
\newcommand{\setcomp}[1]{{#1}^{\mathsf{c}}}

% Create logic command aliases %
\newcommand{\limplies}{\rightarrow}
\newcommand{\nequiv}{\not\equiv}
\newcommand{\liff}{\leftrightarrow}

% first page header & footer %
\fancypagestyle{assignment}{
    \fancyhf{}
    \renewcommand{\footrulewidth}{0.1mm}
    \fancyfoot[R]{\assignmentType\text{ }\assignmentNumber}
    \fancyfoot[C]{\thepage \hspace{1pt} of \pageref{LastPage}}
    \fancyfoot[L]{\courseCode\text{ }\semester}
    \renewcommand{\headrulewidth}{0mm}
}

% Frontmatter for essay page numbering%
\fancypagestyle{frontmatter}{
    \fancyhf{}
    \renewcommand{\footrulewidth}{0.1mm}
    \fancyfoot[R]{\assignmentType\text{ }\assignmentNumber}
    \fancyfoot[C]{\thepage \hspace{1pt} of \pageref{EndFrontMatter}}
    \fancyfoot[L]{\courseCode\text{ }\semester}
    \fancyhead[L]{\name}
    \fancyhead[R]{\studentNumber}
}

% Essay body page numbering %
\fancypagestyle{body}{
    \fancyhf{}
    \renewcommand{\footrulewidth}{0.1mm}
    \fancyfoot[R]{\assignmentType\text{ }\assignmentNumber}
    \fancyfoot[C]{\thepage \hspace{1pt} of \pageref{LastPage}}
    \fancyfoot[L]{\courseCode\text{ }\semester}
    \fancyhead[L]{\name}
    \fancyhead[R]{\studentNumber}
}

% Page header and footers %
\fancyhf{}
\renewcommand{\footrulewidth}{0.1mm}
\fancyfoot[R]{\assignmentType\text{ }\assignmentNumber}
\fancyfoot[C]{\thepage \hspace{1pt} of \pageref{LastPage}}
\fancyfoot[L]{\courseCode\text{ }\semester}
\fancyhead[L]{\name}
\fancyhead[R]{\studentNumber}

% Apply headers & footer page style %
\pagestyle{fancy}

% Assignment first page header %
\newcommand{\beginassignemnt}{
    % Prevent paragraph indents, this isn't an English assignment! %
    \newlength\tindent
    \setlength{\tindent}{\parindent}
    \setlength{\parindent}{0pt}

    \thispagestyle{assignment}
    \noindent
    \courseTitle \hfill \university\\
    \courseCode \hfill \semester
    \begin{center}
        \textbf{\assignmentType\text{ }\ifdefempty{\assignmentNumber}{}{\#}\assignmentNumber}\\
        \name \hspace{1pt} - \studentNumber\\
        \dueDate\\
    \end{center}
    \vspace{6pt}
    \hrule
    \vspace{1.5\headsep}
}

% Essay titlepage stuff %
\newcommand{\beginessay}{
    % Load all citations %
    \nocite{*}

    % Special numbering for front matter %
    \pagestyle{frontmatter}
    \pagenumbering{roman}

    % Titlepage stuff %
    \begin{center}
        \normalsize
        \textsc{\university}\\[5cm]
        \LARGE \textbf{\MakeUppercase{\essayTitle}}\\[0.5cm]
        \large \text{ }\essaySubtitle\text{ }\\[10cm] % blank \texts are used for empty subtitles %
        \normalsize
        \textsc{\name}\\
        \textsc{\studentNumber}\\
        \textsc{\courseCode}\\
        \textsc{\semester}\\
        \textsc{\dueDate}
    \end{center}
    \thispagestyle{empty}
    % End title page stuff %

    % Table of Contents %
    \newpage
    \tableofcontents
    \newpage

    % If more than 1 table/figure show appropriate lists %
    \iftotalfigures
        \addcontentsline{toc}{section}{\listfigurename}
        \listoffigures
    \fi
    \iftotaltables
        \addcontentsline{toc}{section}{\listtablename}
        \listoftables
    \fi

    % Display an abstract if the abstract isn't empty %
    \ifdefempty{\essayAbstract}{}{
        \newpage
        \addcontentsline{toc}{section}{Abstract}
        \begin{abstract}
            \essayAbstract
        \end{abstract}

    }
    \label{EndFrontMatter}
    \newpage

    % Reset page numbering %
    \pagenumbering{arabic}
    \pagestyle{body}
}

% Update the bibliography command to add its self to the references
\let\oldprintbib\printbibliography
\renewcommand{\printbibliography}{
    \newpage
    \oldprintbib
    \addcontentsline{toc}{section}{References}
    \newpage
}

% Begin the document %
\begin{document}

% makes the header for assignment/titlepage for essay %
% \beginessay
\beginassignemnt

\section*{Introduction}

The client wishes to really focus on the detection and prevention of falling. Right now the current solution allows the client to press a single button for all types of issues. The single button is located on their bed and can be difficult to reach, especially after falling. It would be better if there was a specific measure to prevent/detect falling.

\section*{Concepts}

\begin{enumerate}[1)]
    \item
        The first concept is to expand upon the button that is currently in place. The client said we can attach a device to make it easier for the user to press the button. The device would have minimal impact on the bed but would improve functionality of the current system.
    \item
        The second concept is to use a wrist watch which communicates to the nurses using bluetooth. The watch would be able to detect falls easily using the accelerometers while reporting them to the nurse without having to connect to the internet. A watch may be more technically involved for the client though.
    \item
        The final concept is a simple clip-on button which is similar to the watch but only serves one purpose. The idea behind this concept is to be less technically involved. Similar to the hospital's current implementation of one button only this one would be wireless and clip on to the client's shirt.
\end{enumerate}

\section*{Sketches}

Since my sketching skills aren't very good I am writing a few sentences describing each concept in more detail about how they would look.

\begin{enumerate}[1)]
    \item
        The first device would be a 3d printed box. It would clip on to the side of the bed with a large button facing outwards. The button would have on function: pressing the smaller button underneath. This would make it easier for the patient to press the button.
    \item
        The second device would look like a standard digital wrist watch. On the outside there would be 3 buttons to navigate the menus, one which would be used to page a nurse in the case of an emergency. Furthermore the watch would be running software to use its sensors and detect falls.
    \item
        This would be a simple push button. Likely around an 1 inch cubed to make it simple to press. The button its self would provide tactile feedback so the patient is aware when they push it.
\end{enumerate}

\section*{Functionality}

\begin{enumerate}[1)]
    \item
        The first item meets the client's requirement by providing an easy method to contact the nurse should a fall occur getting out of bed. However the specifications also consider the case when the user is further away from their bed thus making it making it more difficult to press the button should the patient be on the other side of the room. It does however make pressing the small button much easier.
    \item
        The second option covers the case where the client is wandering their room/the halls. It also provides the added functionality of waterproofness. The one limitation is that the patients may not be able to use the small buttons of a wrist watch. Furthermore they may not understand all of the technicalities of the watch since they are not exposed to IoT every day.
    \item
        The third concept is similar to the first however it enables the patient to wander around. Thus covering the case where they are not in bed. The problem with this design is it may be an oversimplification since it only allows the user to press the button to notify the nurse in the event of a fall.
\end{enumerate}

\section*{Strategies}

To develop these concepts we first met with the client and had a conversation to determine their needs. I then got creative and generated a few different solutions to fit their needs as best as possible. Their answers motivated my designs in the hopes that it would best fit the specifications generated. Unfortunately I'm terrible at sketching so instead I described the ideas I've got. Based off of the first design I came up with ways to make it more remote so that the patient can travel with the device with out it interfering with their daily life.

\end{document}
